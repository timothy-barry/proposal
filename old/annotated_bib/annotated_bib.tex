\documentclass[12pt]{article}
\usepackage[margin=1in]{geometry}
\usepackage{amsfonts}
\usepackage{amsmath}
\usepackage{graphicx}
\usepackage[dvipsnames]{xcolor}
\usepackage{float}
\usepackage[caption = false]{subfig}
\usepackage{/Users/timbarry/Documents/optionFiles/mymacros}
\newcommand{\red}[1]{\textcolor{red}{#1}}
\newcommand{\green}[1]{\textcolor{ForestGreen}{#1}}
\newcommand{\yellow}[1]{\textcolor{YellowOrange}{#1}}
\usepackage[
backend=biber,
style=authoryear,
maxcitenames=6,
maxbibnames=6,
backref=true,
doi=false,
isbn=false,
url=false,
eprint=false
]{biblatex}
\addbibresource{/Users/timbarry/optionFiles/Proposal.bib}

\begin{document}

\begin{center}
\textbf{Annotated bibliography for proposal} \\
Tim B.
\end{center}

This document is an annotated bibliography of references that might relate to my proposal. All papers are statistical (no biology or computing). The references fall into the following categories: randomization tests, $e$-values, multiple $p$-values, conditional independence tests (non-CRT/CPT), causal inference, and regression with Gaussian response. The references are color-coded: red, yellow, and green indicate high, moderate, and low importance, respectively.

\section{Randomization tests}

Randomization tests refer to tests that recompute a test statistic under a null hypothesis. The most important examples for our purposes are the permutation test, conditional randomization test, and conditional permutation test.

\begin{enumerate}
\item \red{\cite{Katsevich2020b}}.
\item \red{\cite{Berrett2020}}.
\item \red{\cite{Li2021}}.
\item \yellow{\cite{Zhang2021}}.
\item \yellow{\cite{Javanmard2021}}.
\item \yellow{\cite{Zhong2021}}.
\item \green{\cite{Dobriban2021}}.
\end{enumerate}

\section{$e$-values}

\begin{enumerate}
\item \red{\cite{Shafer2021}}.
\item \red{\cite{Wang2020b}}.
\item \red{\cite{Vovk2020}}.
\item 
\end{enumerate}

\section{Multiple testing with $p$-values}

\section{Conditional independence tests (non-CRT/CPT)}

\printbibliography

\end{document}
