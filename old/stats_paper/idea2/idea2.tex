\documentclass[12pt]{article}
\usepackage{amsfonts}
\usepackage{amsmath}
\usepackage{graphicx} 
\usepackage{float}
\usepackage[caption = false]{subfig}
\usepackage{/Users/timbarry/Documents/optionFiles/mymacros}

\begin{document}
\noindent Tim B.
\section{Standard approach}
This document explores a simple strategy for improving the robustness of randomization tests. Let $T_1, \dots, T_{B-1}$ be the $B-1$ resampled test statistics. Let $T^*$ be the statistic computed on the raw data. Let $\textrm{rank}(T^*; B)$ be the rank of $T^*$ in the set $\{T^*, T_1, \dots, T_{B-1}\},$
i.e.
$$ \textrm{rank}(T^*; B) = | \{ T \in \{ T^*, T_1, \dots, T_{B-1} \} : T^* \geq T |  = 1 + \sum_{i=1}^{B-1} \mathbb{I}(T^* \geq T_i).$$
Under the null hypothesis, $(T^*, T_1, \dots, T_{B-1})$ is exchangeable. Therefore,
$$\textrm{rank}(T^*; B) = \textrm{Unif}(\{1, \dots, {B}\}).$$
We can construct a valid $p$-value as follows:
$$p = \textrm{rank}(T^*; B)/B.$$ Letting $T_{(1)}, \dots, T_{(B)}$ denote the order statistics of $T_1, \dots, T_B$, we also can compute the rank of $T^*$ as
$$ \textrm{rank}(T^*; B) = 1 + \sum_{i=1}^{B-1} \mathbb{I}(T^* \geq T_{(i)}),$$ i.e., we can order the $T_i$s before computing the rank.

\section{A simple strategy for improving robustness}
Denote $R := \textrm{rank}(T^*; B)$. Let $f: \{1, \dots, B\} \to \R$ be a function. We consider the transformed random variable $W := f(R)$. Let $\mathcal{X} := \{ f(r) \}_{r=1}^B .$ For $w \in \mathcal{X} $, we have that
$$\P\left(W = w\right) = \P\left( R \in f^{-1}(w) \right) = | f^{-1}(w)|,$$
where $f^{-1}(w) = \{ r \in \{1, \dots, B\} : f(r) = w \}$ is the preimage of $w$. The latter equality follows because $R$ is uniformly distributed on $\{1, \dots, B\}.$ If $f$ is injective, then $P(W = w) = 1/B$ for all $w$, implying that $W$ is uniformly distributed over $\mathcal{X}.$ Therefore,  
\\ \\ \noindent
\textbf{Example}. Let $a_1, \dots, a_{B-1} > 0$ be weights. Denote $a = [a_1, \dots, a_{B-1}]^T \in \R^{B-1}$. Define the \textit{weighted} rank $W$ as follows:
$$W = 1 + \sum_{i=1}^{B-1} a_i \mathbb{I}\left( T^* \geq T_{(i)} \right).$$ Clearly, $W$ is not (in general) uniformly distributed on $\{1, \dots, B\}$, and so $W/B$ is not a valid $p$-value. However, we can compute the distribution of $W$. Define the function $f : \{1, \dots, B\} \to \R$ by $$f(j) = 1 + \sum_{i=1}^{j-1} a_i.$$ Observe that $W = f(R),$ where $$R = \sum_{i=1}^{B-1} \mathbb{I}(T^* \geq T_{(i)})$$ is the (unweighted) rank of $T^*$. Hence, $R$ is uniformly distributed over $\{1, \dots, B\}$, and
$$\P(W = f(j)) = 1/B $$ for $j \in \{1, \dots, B\}.$ 

%Thus, $f^{-1}(W)$ is uniformly distributed over $\{1, \dots, B\}.$ A valid $p$-value is thus
% $$p = f^{-1}(W)/B.$$ 
 
\bibliographystyle{unsrt}
\bibliography{/Users/timbarry/Documents/optionFiles/library.bib}

\end{document}
