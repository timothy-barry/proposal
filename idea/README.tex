\documentclass[12pt]{article}
\usepackage{amsfonts}
\usepackage{amsmath}
\usepackage{graphicx} 
\usepackage{float}
\usepackage[caption = false]{subfig}
\usepackage{/Users/timbarry/Documents/optionFiles/mymacros}

\begin{document}
	
% Basic intuition: the type of value (i.e., $p$-value or $e$-value) that we return from a hypothesis test should match the error criterion that we are using. If we are testing a single hypothesis, the natural type-I error target is the probability of incorrectly rejecting the hypothesis under the null. If we are testing a handful of hypotheses, the natural target is the family-wise error rate. Finally, if we are testing a large number of hypotheses, the natural target is false discovery rate. In the first two cases, $p$-values are the most natural objects for controlling type I error; in the final case, by contrast, $e$-values are most natural.

Consider a permutation test (the ideas here apply broadly to all sampling-based tests of independence or conditional independence, including marginal permutation test, conditional permutation test, marginal randomization test, and conditional randomization test). Let $B$ be the number of resamples. Let $T_{1}, \dots, T_B$ be the ordered resampled test statistics. Let $T^*$ be the test statistic computed on the raw data. Let $I_i$ be the indicator that $T^*$ is less than $T_i$, i.e. $$I_i = \mathbb{I}\left( T^* \leq T_i \right).$$ We have the basic fact under the null hypothesis that
$$\E(I_i) = \P(T^* \leq T_i) = i/B.$$ We want to combine $I_1, \dots, I_B$ in such a way so as to produce an e-value $E$ (i.e., a non-negative variable $E$ such that $\E(E) \leq 1$). To test the hypothesis, we check if $E > \beta$ for some threshold $\beta$.

We call $f: \R^B \to \R^{\geq 0}$ a ``combining function'' if $f$ combines $I_1, \dots, I_B$ in such a way so as to produce a valid E-value, i.e.,
$$f(I_1, \dots, I_B) \geq 0; \quad \E\left[f(I_1, \dots, I_B)\right] \leq 1.$$

Let $\mathcal{F}$ be the class of combining functions. An interesting subset of $\mathcal{F}$ is the set of linear combining functions. A function $g \in \mathcal{F}$ is a linear combining function if
$$ g(I_1, \dots, I_B) = c + \sum_{i=1}^B a_i I_i $$ for given scalars $a_1, \dots, a_B,$ and $c.$ The linearity of expectation implies that
$$ \E\left[ g(I_1, \dots, I_B) \right] = c + \sum_{i=1}^n a_i \E(I_i) = c + \frac{1}{B} \sum_{i=1}^B (a_i)(i).$$ Therefore, under the stricter requirement that $\E(E) = 1$, we have the constraints
\begin{itemize}
\item[i] $\sum_{i=1}^B (a_i) i = B(1-c)$
\item[ii] $c + \sum_{i=1}^B a_i I_i \geq 0.$
\end{itemize}
We give a couple examples of linear combining functions. 
\\ \\
\noindent
\textbf{Example 1}. Let $c = 0$ and $a_i = 1/i$ for all $i$. Then 
$$ \sum_{i=1}^B (1/i) i = B(1 - 0),$$ confirming constraint (i). Next, $$ 0 + \sum_{i=1}^B \frac{I_i}{i} \geq 0,$$ confirming constraint (ii). Therefore,
$$ E = \sum_{i=1}^B \frac{I_i}{i}$$ is a valid $e$-value. This is a left-tailed test: if $T^*$ is small, then $E$ is big, leading us to reject the null.
\\ \\
\noindent
\textbf{Example 2}. The standard permutation test is a special case of the proposed framework. Let $c = 3/2 + 1/(2B)$ and $a_i = -1/B$ for all $a_i$. Verifying condition (i),
\begin{multline*}
\sum_{i=1}^B (a_i)i = - \frac{1}{B} \sum_{i=1}^B i =  -\frac{1}{B} \left( \frac{B^2 + B}{2} \right) = -(1/2)B - 1/2 \\ = B(1 - (3/2) - 1/(2B)) = B(1-c).
\end{multline*}
Next, verifying condition (ii),
$$ (3/2) + 1/(2B) -(1/B) \sum_{i=1}^B I_i \geq (3/2) +1/(2B) - 1 \geq 0.$$
Therefore,
$$E =  3/2 + 1/(2B) - \frac{1}{B}\sum_{i=1}^B I_i$$ is a valid $e$-value. Note that $p_B := \frac{1}{B} \sum_{i=1}^B I_i$ is simply the $p$-value corresponding to the right-sided permutation test. For given $\alpha \in (0,1)$, we reject the null hypothesis if and only if $p_B < \alpha$, which is equivalent to rejecting the null hypothesis if and only if 
$$ E > 3/2 + 1/(2B) - \alpha.$$
Therefore, the $p$-permutation test is a special case of the $e$-permutation test. We typically think of $e$-values as controlling FDR, but here $e$-values control type-I error.

We can extend this idea to control family-wise error rate. 

\noindent
\\ \\
\textbf{Example 3}. Suppose we are using CRT or CPT to test conditional independence. If we have misspecified the model for $X|Z$, then the resampled test statistics $T_1, \dots, T_B$ will be off. In particular, qq-plots often reveal that in such settings  the tail of the distribution is incorrect, leading to $p$-value inflation. We can use the proposed framework to downweight re-sampled test statistics in the tail of the distribution, possibly leading to better calibration. For example, consider again example 1, where previously we set $a_i = 1/i$ for all $i$. Suppose that we use (e.g.) a Gaussian kernel to assign the weights. Let
$$\hat{\mu} = \frac{1}{B} \sum_{i=1}^B T_i$$ be the average of the resampled test statistics. Next, let $$K(t) = \frac{1}{\sqrt{2 \pi}} e^{-(1/2)t^2}$$ be the Gaussian kernel. Define $a_i$ as
$$a_i :=(B/i)\frac{K(|T_i - \hat{\mu}|/h) }{\sum_{i=1}^B K(|T_i - \hat{\mu}|/h)  }.$$ Again, set $c = 0$. We see that
$$ \sum_{i=1}^B (a_i) i = \frac{ \sum_{i=1}^B K(|T_i - \hat{\mu}|/h) }{ \sum_{i=1}^B K(|T_i - \hat{\mu}|/h)} = B = B(1 - c),$$ verifying the first condition. Moreover, $A_i \geq 0$ for all $i$, satisfying the second condition. It follows that $$E  := \sum_{i=1}^B a_i I_i$$ is a valid $e$-value. This $e$-value is  less sensitive to the tails of the sampling distribution than the $e$-value from example 1.

\textbf{Advantages to this approach}. The proposed framework has several advantages (I think). First, because the outputs are $e$-values, they can be combined under arbitrary dependence to control FDR using the $e$-BH procedure. Second, the framework is flexible: we can choose among many different ``combining functions'' (including the easy-to-use linear variants) to achieve different objectives. Certain combining functions might have robustness properties, for example. Third, we easily can combine e-values across data splits, enabling us to run multiple train-test splits without worrying about combining the resulting p-values.

\textbf{Open questions}. What is the ``best'' combining function that we could use? 

\textbf{Additional properties of the $I_i$s}. Recall that $I_i = \mathbb{I}(T^* \leq T_i).$ We state several additional properties of the $I_i$s. Let $k_1, k_d, \dots, k_p \in \N$ be indexes such that
$$ 1 \leq k_1 \leq k_2 \leq \dots \leq k_p \leq B.$$ Furthermore, let $r_1, r_2, \dots, r_p \in \N$ be non-negative integers. We have that
$$ I_{k_1}^{r_1} I_{k_2}^{r_2} \dots I_{k_p}^{r_p} = I_{k_1}.$$ The reason is as follows. First, because $I_{k_i}$ is Bernoulli,
$$  I_{k_1}^{r_1} I_{k_2}^{r_2} \dots I_{k_p}^{r_p} = I_{k_1} I_{k_2} \dots, I_{k_p}.$$ Next, we take cases on $T^*$. First, suppose that $T^* \leq k_1$. Then $T^* \leq k_2, \dots, k_p.$ Therefore,
$$I_{k_1} I_{k_2} \dots I_{k_p} = 1 \cdot 1 \dots 1 = I_{k_1}.$$ Next, suppose that $T^* > k_1$. Then $I_{k_1} = 0,$ and so
$$ I_{k_1} I_{k_2} \dots I_{k_p} = 0 \cdot I_{k_2} \dots I_{k_p} = 0 = I_{k_1}.$$
This property is useful because it allows us to more easily evaluate expressions of the form
$\left(\sum_{i=1}^n I_i \right)^r$ for $r \in \N.$ As a warmup, we consider $r = 1$:
$$ \sum_{i=1}^B I_i = a^{(1)}_i I_i,$$ where $a^{(1)}_1 = 1.$ Next, consider $r = 2$:
\begin{multline*}
\left( \sum_{i=1}^B I_i \right)^2 = \sum_{i=1} ^B\sum_{j=1}^B I_i I_j  = \sum_{i=1}^B I_i^2 + 2 \sum_{i = 1}^{B-1} \sum_{j=i+1}^{B} I_i I_j =  \sum_{i=1}^B I_i + 2 \sum_{i=1}^{B-1}  \sum_{j=i+1}^{B} I_i \\ = \sum_{i=1}^B I_i + \sum_{i=1}^{B-1} 2 (B-i) I_i = \sum_{i=1}^B I_i + \sum_{i=1}^B 2(B - i)I_i = \sum_{i=1}^B (2B - 2i + 1)I_i = \sum_{i=1}^B a^{(2)}_i I_i,
\end{multline*} 
where $a^{(2)}_i = 2B - 2i + 1$.
%\begin{multline*}
%\left( \sum_{i=1}^B I_i \right)^3 = \sum_{i=1}^B \sum_{j=1}^B \sum_{k=1}^B I_i I_j I_k = 
% \end{multline*}

Consider the multinomial theorem: 
$$ \left( \sum_{i=1}^B I_i \right)^r = \sum_{k_1 + k_2 + \dots + k_B = r} \binom{ r }{ k_1, k_2, \dots, k_B } \prod_{i=1}^B I_{i}^{k_i},$$ where $$ \binom{r}{k_1, k_2, \dots, k_B} = \frac{r!}{k_1! k_2! \dots k_B!}.$$
As we have shown previously, for given $k_1, \dots, k_B \in \{0, 1, \dots, r\},$
$$ \prod_{i=1}^B I_i^{k_i} = I_{\tau(k_1, \dots, k_B)},$$ where $$\tau(k_1, \dots, k_B) = \min\{i : \{1, \dots, B\} : k_i \neq 0 \}.$$ Thus, $\tau : \{0, 1, \dots, r\}^B \to \{1, \dots, B\}$ is function that returns the $I_i$ of a given index configuration. In words $\tau$ gives the minimal nonzero index. We also need to determine $\tau^{-1}(s)$, i.e.\ the set of configurations $(k_1, \dots, k_B)$ such that $\tau(k_1, \dots, k_B) = s.$ It is clear that
$$ \tau^{-1}(s) = \{ (k_1, k_2, \dots, k_B) : k_1 = k_2 = \dots = k_{s-1} = 0, k_s + k_{s+1} + \dots + k_B = r\}.$$
Therefore, the multinomial theorem reduces to
$$\left( \sum_{i=1}^B I_i \right)^r = \sum_{k_1 + k_2 + \dots + k_B = r} \binom{ r }{ k_1, k_2, \dots, k_B } I_{\tau(k_1, \dots, k_B)}.$$ We know that the outcome is a linear combination of the $I_i$s. Therefore, we can write
$$ \left( \sum_{i=1}^B I_i \right)^r = \sum_{s=1}^B \sum_{ \tau^{-1}(s)} \binom{r}{ k_1, k_2, \dots, k_B } I_s.$$ However, $\tau^{-1}(s) = \{(0, 0, 0, \dots, 0, k_s, k_{s+1}, \dots, k_B) : k_s + k_{s+1} + \dots + k_B = r\}.$ Therefore,
\begin{multline*}
\sum_{\tau^{-1}(s)} \binom{r}{k_1, k_2, \dots, k_B} = \sum_{k_s + k_{s+1} + \dots + k_B = r}  \binom{r}{k_s, k_{s+1}, \dots, k_B} \\ = \sum_{k_1 + k_2 + \dots + k_{B-s + 1} = r} \binom{r}{ k_1, k_2, \dots, k_{B-s+1}} := M(r,B,s).
\end{multline*}
We conclude that
$$ \left( \sum_{s=1}^B I_s \right)^r = \sum_{s=1}^B M(r,B,s) I_s.$$
We evaluate this for $r=1$ and $r=2$ to confirm correctness. First, setting $r=1$, we have that $M(1,B,s) = 1$. Therefore,
$$ \sum_{i=1}^B I_i = \sum_{s=1}^B I_s.$$

%We consider the special case of $r = 1$:
%$$
%\left(\sum_{i=1}^B I_i \right)^1 = \sum_{k_1 + k_2 + \dots + k_B = 1} \binom{1}{k_1, k_2, \dots, k_B} I_{\tau(k_1, \dots, k_B)} = \sum_{i=1}^B I_i,
%$$ and so $a_i^{(i)} = 1$. Next, consider $r=2$:
%$$
%\left( \sum_{i=1}^B I_i \right)^2 = \sum_{k_1 + k_2 + \dots + k_B = 2} \binom{1}{k_1, k_2, \dots, k_B} I_{\tau(k_1, \dots, k_B)}.
%$$
%We know that the outcome is a linear combinations of $I_i$s. Therefore, we focus our attention on $I_s$ for $s \in \{1, 2, \dots, B\}$. We have that $$\tau^{-1}(s) =  $$


\end{document}
